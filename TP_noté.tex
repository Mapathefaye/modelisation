% Options for packages loaded elsewhere
\PassOptionsToPackage{unicode}{hyperref}
\PassOptionsToPackage{hyphens}{url}
%
\documentclass[
]{article}
\usepackage{amsmath,amssymb}
\usepackage{iftex}
\ifPDFTeX
  \usepackage[T1]{fontenc}
  \usepackage[utf8]{inputenc}
  \usepackage{textcomp} % provide euro and other symbols
\else % if luatex or xetex
  \usepackage{unicode-math} % this also loads fontspec
  \defaultfontfeatures{Scale=MatchLowercase}
  \defaultfontfeatures[\rmfamily]{Ligatures=TeX,Scale=1}
\fi
\usepackage{lmodern}
\ifPDFTeX\else
  % xetex/luatex font selection
\fi
% Use upquote if available, for straight quotes in verbatim environments
\IfFileExists{upquote.sty}{\usepackage{upquote}}{}
\IfFileExists{microtype.sty}{% use microtype if available
  \usepackage[]{microtype}
  \UseMicrotypeSet[protrusion]{basicmath} % disable protrusion for tt fonts
}{}
\makeatletter
\@ifundefined{KOMAClassName}{% if non-KOMA class
  \IfFileExists{parskip.sty}{%
    \usepackage{parskip}
  }{% else
    \setlength{\parindent}{0pt}
    \setlength{\parskip}{6pt plus 2pt minus 1pt}}
}{% if KOMA class
  \KOMAoptions{parskip=half}}
\makeatother
\usepackage{xcolor}
\usepackage[margin=1in]{geometry}
\usepackage{color}
\usepackage{fancyvrb}
\newcommand{\VerbBar}{|}
\newcommand{\VERB}{\Verb[commandchars=\\\{\}]}
\DefineVerbatimEnvironment{Highlighting}{Verbatim}{commandchars=\\\{\}}
% Add ',fontsize=\small' for more characters per line
\usepackage{framed}
\definecolor{shadecolor}{RGB}{248,248,248}
\newenvironment{Shaded}{\begin{snugshade}}{\end{snugshade}}
\newcommand{\AlertTok}[1]{\textcolor[rgb]{0.94,0.16,0.16}{#1}}
\newcommand{\AnnotationTok}[1]{\textcolor[rgb]{0.56,0.35,0.01}{\textbf{\textit{#1}}}}
\newcommand{\AttributeTok}[1]{\textcolor[rgb]{0.13,0.29,0.53}{#1}}
\newcommand{\BaseNTok}[1]{\textcolor[rgb]{0.00,0.00,0.81}{#1}}
\newcommand{\BuiltInTok}[1]{#1}
\newcommand{\CharTok}[1]{\textcolor[rgb]{0.31,0.60,0.02}{#1}}
\newcommand{\CommentTok}[1]{\textcolor[rgb]{0.56,0.35,0.01}{\textit{#1}}}
\newcommand{\CommentVarTok}[1]{\textcolor[rgb]{0.56,0.35,0.01}{\textbf{\textit{#1}}}}
\newcommand{\ConstantTok}[1]{\textcolor[rgb]{0.56,0.35,0.01}{#1}}
\newcommand{\ControlFlowTok}[1]{\textcolor[rgb]{0.13,0.29,0.53}{\textbf{#1}}}
\newcommand{\DataTypeTok}[1]{\textcolor[rgb]{0.13,0.29,0.53}{#1}}
\newcommand{\DecValTok}[1]{\textcolor[rgb]{0.00,0.00,0.81}{#1}}
\newcommand{\DocumentationTok}[1]{\textcolor[rgb]{0.56,0.35,0.01}{\textbf{\textit{#1}}}}
\newcommand{\ErrorTok}[1]{\textcolor[rgb]{0.64,0.00,0.00}{\textbf{#1}}}
\newcommand{\ExtensionTok}[1]{#1}
\newcommand{\FloatTok}[1]{\textcolor[rgb]{0.00,0.00,0.81}{#1}}
\newcommand{\FunctionTok}[1]{\textcolor[rgb]{0.13,0.29,0.53}{\textbf{#1}}}
\newcommand{\ImportTok}[1]{#1}
\newcommand{\InformationTok}[1]{\textcolor[rgb]{0.56,0.35,0.01}{\textbf{\textit{#1}}}}
\newcommand{\KeywordTok}[1]{\textcolor[rgb]{0.13,0.29,0.53}{\textbf{#1}}}
\newcommand{\NormalTok}[1]{#1}
\newcommand{\OperatorTok}[1]{\textcolor[rgb]{0.81,0.36,0.00}{\textbf{#1}}}
\newcommand{\OtherTok}[1]{\textcolor[rgb]{0.56,0.35,0.01}{#1}}
\newcommand{\PreprocessorTok}[1]{\textcolor[rgb]{0.56,0.35,0.01}{\textit{#1}}}
\newcommand{\RegionMarkerTok}[1]{#1}
\newcommand{\SpecialCharTok}[1]{\textcolor[rgb]{0.81,0.36,0.00}{\textbf{#1}}}
\newcommand{\SpecialStringTok}[1]{\textcolor[rgb]{0.31,0.60,0.02}{#1}}
\newcommand{\StringTok}[1]{\textcolor[rgb]{0.31,0.60,0.02}{#1}}
\newcommand{\VariableTok}[1]{\textcolor[rgb]{0.00,0.00,0.00}{#1}}
\newcommand{\VerbatimStringTok}[1]{\textcolor[rgb]{0.31,0.60,0.02}{#1}}
\newcommand{\WarningTok}[1]{\textcolor[rgb]{0.56,0.35,0.01}{\textbf{\textit{#1}}}}
\usepackage{graphicx}
\makeatletter
\def\maxwidth{\ifdim\Gin@nat@width>\linewidth\linewidth\else\Gin@nat@width\fi}
\def\maxheight{\ifdim\Gin@nat@height>\textheight\textheight\else\Gin@nat@height\fi}
\makeatother
% Scale images if necessary, so that they will not overflow the page
% margins by default, and it is still possible to overwrite the defaults
% using explicit options in \includegraphics[width, height, ...]{}
\setkeys{Gin}{width=\maxwidth,height=\maxheight,keepaspectratio}
% Set default figure placement to htbp
\makeatletter
\def\fps@figure{htbp}
\makeatother
\setlength{\emergencystretch}{3em} % prevent overfull lines
\providecommand{\tightlist}{%
  \setlength{\itemsep}{0pt}\setlength{\parskip}{0pt}}
\setcounter{secnumdepth}{-\maxdimen} % remove section numbering
\ifLuaTeX
  \usepackage{selnolig}  % disable illegal ligatures
\fi
\IfFileExists{bookmark.sty}{\usepackage{bookmark}}{\usepackage{hyperref}}
\IfFileExists{xurl.sty}{\usepackage{xurl}}{} % add URL line breaks if available
\urlstyle{same}
\hypersetup{
  pdftitle={Tp\_Noté},
  pdfauthor={Mapathé FAYE},
  hidelinks,
  pdfcreator={LaTeX via pandoc}}

\title{Tp\_Noté}
\author{Mapathé FAYE}
\date{2024-04-04}

\begin{document}
\maketitle

\hypertarget{r-markdown}{%
\subsection{R Markdown}\label{r-markdown}}

This is an R Markdown document. Markdown is a simple formatting syntax
for authoring HTML, PDF, and MS Word documents. For more details on
using R Markdown see \url{http://rmarkdown.rstudio.com}.

When you click the \textbf{Knit} button a document will be generated
that includes both content as well as the output of any embedded R code
chunks within the document. You can embed an R code chunk like this:

\begin{Shaded}
\begin{Highlighting}[]
\CommentTok{\# Chargement des données }
\NormalTok{poules }\OtherTok{\textless{}{-}} \FunctionTok{read.table}\NormalTok{(}\AttributeTok{file =} \StringTok{"poules.txt"}\NormalTok{, }\AttributeTok{header =}\NormalTok{ T)}
\CommentTok{\# Afficher les premières ligne du jeu de données}
\FunctionTok{head}\NormalTok{(poules)}
\end{Highlighting}
\end{Shaded}

\begin{verbatim}
##   id genotype Temperature genT PREC pav_ab  p_ga p_coeur p_foie p_sang p_plum
## 1  1       FF          32 FF32  153   2103 182.9     5.4   24.1     74     41
## 2  2       NN          32 ff32  146   1719  75.0     4.9   18.0     57    101
## 3  3       FN          32 fF32  146   2070 133.3     4.7   20.0     44    104
## 4  4       FN          32 fF32  146   1976 113.9     4.7   21.8     49     75
## 5  5       FF          32 FF32  141   1598  88.4     4.7   21.9     52      7
## 6  6       NN          32 ff32  145   2074 138.4     4.6   29.8     53    122
\end{verbatim}

\begin{Shaded}
\begin{Highlighting}[]
\CommentTok{\# résumé statistiques des données numériques}
\FunctionTok{summary}\NormalTok{(poules)}
\end{Highlighting}
\end{Shaded}

\begin{verbatim}
##        id           genotype          Temperature        genT          
##  Min.   :  1.00   Length:108         Min.   :22.00   Length:108        
##  1st Qu.: 27.75   Class :character   1st Qu.:22.00   Class :character  
##  Median : 54.50   Mode  :character   Median :22.00   Mode  :character  
##  Mean   : 54.50                      Mean   :26.72                     
##  3rd Qu.: 81.25                      3rd Qu.:32.00                     
##  Max.   :108.00                      Max.   :32.00                     
##       PREC           pav_ab          p_ga          p_coeur          p_foie     
##  Min.   :135.0   Min.   :1515   Min.   : 53.1   Min.   :3.100   Min.   :16.10  
##  1st Qu.:145.0   1st Qu.:1852   1st Qu.:111.4   1st Qu.:4.700   1st Qu.:24.45  
##  Median :147.0   Median :2064   Median :139.7   Median :5.200   Median :29.35  
##  Mean   :147.5   Mean   :2073   Mean   :145.2   Mean   :5.313   Mean   :33.42  
##  3rd Qu.:150.0   3rd Qu.:2259   3rd Qu.:172.1   3rd Qu.:5.900   3rd Qu.:37.08  
##  Max.   :160.0   Max.   :3135   Max.   :330.3   Max.   :7.900   Max.   :84.60  
##      p_sang          p_plum      
##  Min.   :31.00   Min.   :  7.00  
##  1st Qu.:51.00   1st Qu.: 31.00  
##  Median :60.50   Median : 83.50  
##  Mean   :60.26   Mean   : 71.31  
##  3rd Qu.:69.00   3rd Qu.:100.25  
##  Max.   :92.00   Max.   :150.00
\end{verbatim}

\begin{Shaded}
\begin{Highlighting}[]
\CommentTok{\# afficher informations sur les variables}
\FunctionTok{str}\NormalTok{(poules)}
\end{Highlighting}
\end{Shaded}

\begin{verbatim}
## 'data.frame':    108 obs. of  11 variables:
##  $ id         : int  1 2 3 4 5 6 7 8 9 10 ...
##  $ genotype   : chr  "FF" "NN" "FN" "FN" ...
##  $ Temperature: int  32 32 32 32 32 32 32 32 32 32 ...
##  $ genT       : chr  "FF32" "ff32" "fF32" "fF32" ...
##  $ PREC       : int  153 146 146 146 141 145 142 145 158 149 ...
##  $ pav_ab     : int  2103 1719 2070 1976 1598 2074 2065 1787 2114 2218 ...
##  $ p_ga       : num  182.9 75 133.3 113.9 88.4 ...
##  $ p_coeur    : num  5.4 4.9 4.7 4.7 4.7 4.6 4.1 5 5.2 5.2 ...
##  $ p_foie     : num  24.1 18 20 21.8 21.9 29.8 37.7 24.6 45.6 57.2 ...
##  $ p_sang     : int  74 57 44 49 52 53 36 63 48 67 ...
##  $ p_plum     : int  41 101 104 75 7 122 112 31 105 81 ...
\end{verbatim}

\hypertarget{including-plots}{%
\subsection{Including Plots}\label{including-plots}}

You can also embed plots, for example:

\includegraphics{TP_noté_files/figure-latex/pressure-1.pdf}

Note that the \texttt{echo\ =\ FALSE} parameter was added to the code
chunk to prevent printing of the R code that generated the plot.

\end{document}
